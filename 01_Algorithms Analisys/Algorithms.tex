\documentclass{subfiles}[../Book.tex]

\begin{document}
	\chapter{Introducción a los Algoritmos}
	\section{Introducción}
	Un algoritmo es un conjunto finito no ambiguo de conjunto de instrucciones para resolver un problema dado. Los conocimientos de los algoritmos nos ayudan a obtener resultados deseados mas rápidos, aplicando el algoritmo apropiado. Aprendemos de la experiancia. Con la experiencia, es más sencillo resolver nuevos problemas. Por observación dentro de varios problemas resueltos de algoritmos o técnicas, comenxamos a desarrollar algún patron que puede ayudarnos en resolviendo problemas similares.\\
	
	Las propiedades de un algoritmo son:
	\begin{enumerate}
		\item Tomán cero o más entradas.
		\item Debe de producir una o más salidas.
		\item Debe ser determinista. Esto produce la misma salida si la entrada es la misma también.
		\item Debe de ser correcto. Un programa debe de ser correcto y capas de procesar las entradas dadas si provee la salida correcta.
		\item Debe ser Terminado en un tiempo finito.
		\item  Debe ser eficiente. El algoritmo debe ser eficiente solucionando problemas.
	\end{enumerate}
	La complejidad de un algoritmo es la cantiad de tiempo o espacio requerido por el algoritmo para procesar la entrada y producir una salida.\\
	
	Hay dos tipos de complejidad:
	\begin{enumerate}
		\item  La primera es Complejidad-Tiempo \textbf{Complejidad en Tiempo}, cuanto tiempo es requerido por un algorimto para producir una salida para una entrada de tamaño \textbf{"n"}. la complejidad-tiempo es representada por una función $T(n)$ - tiempo requerido contra la entrada "$n$".
		
	\end{enumerate}
\end{document}